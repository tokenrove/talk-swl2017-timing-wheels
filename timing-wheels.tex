\documentclass{beamer}
\usepackage[utf8]{inputenc}
\usepackage[english]{babel}
\usepackage{hyperref}
\usepackage{minted}
%%\usemintedstyle{perldoc}
\usetheme{default}
\beamertemplatenavigationsymbolsempty{}

\title{Implementations of Timing Wheels}
\author{Julian Squires}
\institute{AdGear Technologies, Inc.}
\date{March 16, 2017}
\begin{document}

\begin{frame}
  \titlepage{}
\end{frame}

\begin{frame}
  Varghese and Lauck. ``Hashed and hierarchical timing
wheels: Data structures for the efficient implementation of a timer
facility.'' ACM SIGOPS Operating Systems Review 21.5 (1987): 25--38.

\vspace{1.5em}

Varghese and Lauck. ``Hashed and hierarchical timing wheels: efficient
data structures for implementing a timer facility.''  IEEE/ACM
transactions on networking 5.6 (1997): 824--834.

\vspace{1.5em}

Varghese. Network Algorithmics: An Interdisciplinary Approach
To Designing Fast Networked Devices. Morgan Kaufmann, 2004.
\end{frame}

\begin{frame}{Recent activity}
  \begin{itemize}
  \item LWN on Gleixner's timing wheels patch:
    \hspace{1em}\hyperlink{https://lwn.net/Articles/646950/}{https://lwn.net/Articles/646950/}

  \item Adrian Colyer's Morning Paper blog:
    \hspace{1em}\hyperlink{https://blog.acolyer.org/2015/11/23/hashed-and-hierarchical-timing-wheels/}{https://blog.acolyer.org/2015/11/23/hashed-and-hierarchical-timing-wheels/}

  \item Juho Snellman on Ratas:
    \hspace{1em}\hyperlink{https://www.snellman.net/blog/archive/2016-07-27-ratas-hierarchical-timer-wheel/}{https://www.snellman.net/blog/archive/2016-07-27-ratas-hierarchical-timer-wheel/}
  \end{itemize}
\end{frame}

\begin{frame}{Motivation}
  \begin{columns}
    \column{0.5\textwidth}\centering
    Optimistic\\[.2cm]
    \begin{itemize}
      \item simulations
      \item scheduling
      \item flow control
      \item circuit breakers
      \item watchdogs
    \end{itemize}
    \column{0.5\textwidth}\centering
    Pessimistic\\[.2cm]
    \begin{itemize}
    \item best-effort requests
    \item TCP retransmit timers
    \item IO timeouts
    \end{itemize}
  \end{columns}
\end{frame}

\begin{frame}{Guarantees of a timer system}
  A timer scheduled to fire after $t$ ticks will have its action
  executed, some time after $t$ ticks (if your clock is monotonic and
  doesn't jump forward or skew forward and \ldots)
\end{frame}

\begin{frame}[t,fragile]
  \frametitle{Zephyr: schedule (\texttt{kernel/include/timeout\_q.h})}

\begin{minted}{c}
static inline void _add_timeout([...])
{
    [...]
    SYS_DLIST_FOR_EACH_NODE(&_timeout_q, node) {
        struct _timeout *in_q = (struct _timeout *)node;

        if (*delta <= in_q->delta_ticks_from_prev) {
            in_q->delta_ticks_from_prev -= *delta;
            sys_dlist_insert_before(&_timeout_q, node,
                                    &timeout->node);
            goto inserted;
        }

        *delta -= in_q->delta_ticks_from_prev;
    }
    sys_dlist_append(&_timeout_q, &timeout->node);
    [...]
}
\end{minted}
\end{frame}

\end{document}
